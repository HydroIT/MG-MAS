%% Generated by Sphinx.
\def\sphinxdocclass{report}
\documentclass[letterpaper,10pt,english]{sphinxmanual}
\ifdefined\pdfpxdimen
   \let\sphinxpxdimen\pdfpxdimen\else\newdimen\sphinxpxdimen
\fi \sphinxpxdimen=49336sp\relax

\usepackage[margin=1in,marginparwidth=0.5in]{geometry}
\usepackage[utf8]{inputenc}
\ifdefined\DeclareUnicodeCharacter
  \DeclareUnicodeCharacter{00A0}{\nobreakspace}
\fi
\usepackage{cmap}
\usepackage[T1]{fontenc}
\usepackage{amsmath,amssymb,amstext}
\usepackage{babel}
\usepackage{times}
\usepackage[Bjarne]{fncychap}
\usepackage{longtable}
\usepackage{sphinx}

\usepackage{multirow}
\usepackage{eqparbox}

% Include hyperref last.
\usepackage{hyperref}
% Fix anchor placement for figures with captions.
\usepackage{hypcap}% it must be loaded after hyperref.
% Set up styles of URL: it should be placed after hyperref.
\urlstyle{same}

\addto\captionsenglish{\renewcommand{\figurename}{Fig.\@ }}
\addto\captionsenglish{\renewcommand{\tablename}{Table }}
\addto\captionsenglish{\renewcommand{\literalblockname}{Listing }}

\addto\extrasenglish{\def\pageautorefname{page}}

\setcounter{tocdepth}{1}



\title{MG-MAS Documentation}
\date{Dec 12, 2016}
\release{}
\author{Idan Tene, Lukas Vozda, Nikolaus Risslegger}
\newcommand{\sphinxlogo}{}
\renewcommand{\releasename}{Release}
\makeindex

\begin{document}

\maketitle
\sphinxtableofcontents
\phantomsection\label{index::doc}

\begin{quote}
\phantomsection\label{goals:goals}\end{quote}


\chapter{Project Goals}
\label{goals::doc}\label{goals:mg-mas-s-documentation}\label{goals:goals}\label{goals:project-goals}
Plan A: write a program which generate awesome music, get famous, get rich, don't have to work the entire life

Plan B: at least 5 credits.


\section{General Info}
\label{goals:general-info}
MG-MAS (music generation - multi agent system) is developed as a project for the Computational Creativity and Multi-Agent Systems lecture in fall 2016. The developers are international students who are taking a stay abroad at the University of Helsinki:
\begin{itemize}
\item {} 
\href{mailto:tene@helsinki.fi}{Idan Tene}

\item {} 
\href{mailto:vozda@helsinki.fi}{Lukas Vozda}

\item {} 
\href{mailto:risslegg@helsinki.fi}{Nikolaus Risslegger}

\end{itemize}

The project tries to generate pleasent sounding pieces of music, therefore it analyzes existing tracks, generates
Markov chains and produce a MIDI file:

\begin{sphinxVerbatim}[commandchars=\\\{\}]
\PYG{n}{mg\PYGZus{}mas}\PYG{o}{.}\PYG{n}{py}
\end{sphinxVerbatim}

Since the variant using Markov chains was not generating meaningful output a second attempt works with neural networks:

\begin{sphinxVerbatim}[commandchars=\\\{\}]
\PYG{n}{midi\PYGZus{}gen\PYGZus{}list}\PYG{o}{.}\PYG{n}{py}
\end{sphinxVerbatim}
\begin{quote}
\phantomsection\label{getting_started:getting-started}\end{quote}


\chapter{Getting started}
\label{getting_started:getting-started}\label{getting_started::doc}\label{getting_started:id1}
The programs should be compiled with python 3.5.
It was tested on three different machines (ubuntu 16.06, Windows 10 and MacOX)

The projects consists of two independet implementations:
\begin{itemize}
\item {} 
\sphinxstylestrong{mg\_mas.py} - calcuating the probabilities with markov chains based on the implentation of the first lecture classes.

\item {} 
\sphinxstylestrong{midi\_gen\_stm.py} - using neural networks instead

\end{itemize}


\section{Dependencies}
\label{getting_started:dependencies}
Following python modules (and their dependencies) must be installed
\begin{itemize}
\item {} 
music21 (\url{http://web.mit.edu/music21/})

\item {} 
numpy (\url{http://www.numpy.org/})

\item {} 
scipy

\item {} 
theano

\item {} 
keras

\item {} 
h5py

\item {} 
tensorflow

\end{itemize}

You may install them with:

\begin{sphinxVerbatim}[commandchars=\\\{\}]
\PYG{n}{python3} \PYG{o}{\PYGZhy{}}\PYG{n}{m} \PYG{n}{pip} \PYG{n}{install} \PYG{n}{package}
\end{sphinxVerbatim}


\section{Running the program}
\label{getting_started:running-the-program}

\chapter{Indices and tables}
\label{index:indices-and-tables}\begin{itemize}
\item {} 
\DUrole{xref,std,std-ref}{genindex}

\item {} 
\DUrole{xref,std,std-ref}{search}

\end{itemize}



\renewcommand{\indexname}{Index}
\printindex
\end{document}